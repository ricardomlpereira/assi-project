\section{Introdução}
\IEEEPARstart{N}{uma} sociedade contemporânea que apresenta um aumento progressivo da sua interconetividade, os ataques \textit{Distributed Denial-of-Service} (DDoS) prosseguem a representar uma das ameaças mais significativas, nomeadamente no âmbito destrutivo, ao quadro global da cibersegurança.

Neste seguimento, cabe notar que um ataque DDoS consiste, essencialmente, numa tentativa maliciosa
de causar disrupção, por vezes total, na disponibilidade de um sistema alvo, nomeadamente servidores, serviços e redes, mediante a utilização de processos e ferramentas que possibilitam a sobrecarga deste mesmo ou da sua infraestrutura adjacente com uma desproporcionada enchente de tráfego irrelevante \cite{cloudflare_what_is_ddos,ibm_what_is_ddos}. Apesar destes ataques serem abrangidos pela categoria de ofensivas de \textit{Denial-of-Service} (DoS), estes distinguem-se dos demais, principalmente, pela sua natureza distribuída, ou seja, mediante a utilização de diversos dispositivos eletrónicos, principalmente computadores, dispositivos \textit{Internet of Things} (IoT) e outros aparelhos que possuam ligações à Internet, de modo a orquestrar um ataque coordenado que almeja promover no seu alvo um grau de inacessibilidade significativo, prejudicando, consequentemente, os seus utilizadores legítimos \cite{zenamor_differences_dos_and_ddos}.

Assim, importa realçar que a constituição de uma rede de dispositivos eletrónicos que possam ser empregues numa ofensiva DDoS representa a generalidade de ataques deste tipo, sendo esta, frequentemente, realizada mediante a infeção de sistemas vulneráveis com diferentes categorias de \textit{malware}. Estes sistemas, uma vez comprometidos, podem ser controlados remotamente pelo atacante, passando a ser designados como \textit{bots} e, consequentemente, integrando uma \textit{robot network} (BOTNET), podendo ser empregues pelo atacante como uma fonte de tráfego irrelevante. Nesta senda, cabe notar que a constituição de uma BOTNET possibilita que o atacante aumente a capacidade disruptiva do seu ataque, assim como dificulte a identificação da origem deste, uma vez que, para o sistema alvo, mais especificamente, os seus processos e ferramentas de proteção contra ataques DDoS, existem inúmeras fontes de tráfego distintas, minimizando a capacidade destes últimos localizarem com elevado grau de precisão o dispositivo coordenador do ataque \cite{cloudflare_what_is_ddos}.

A proteção dos sistemas supramencionados contra ataques DDoS revela-se imperativa, de modo a assegurar, principalmente, a sua disponibilidade, mas, também, a sua integridade. Os impactos de um ataque DDoS bem-sucedido numa organização podem afetar diversos âmbitos distintos desta, nomeadamente a sua reputação, os seus recursos financeiros, assim como a sua relação com os seus próprios funcionários e com os seus clientes, já que estes podem se encontrar privados, respetivamente, de realizarem as suas tarefas laborais e de acederem aos serviços prestados pela empresa \cite{kaspersky_how_ddos_works}.


Importa realçar que existe uma ampla variedade de indivíduos e entidades empresariais envolvidas em ataques DDoS, mais especificamente no horizonte ofensivo destes, assim como no defensivo. Deste modo, as motivações empregues na execução de ataques DDoS são, também, vastas, complicando, consequentemente, a identificação de intuitos concretos que possam ser associados a estes. Contudo, é possível reconhecer que, frequentemente, os ataques DDoS são realizados visando obter ganhos financeiros ou de causar o máximo de disrupção possível aos sistemas alvo, de modo a marcar posições e a viabilizar ações de \textit{hackvism} \cite{fortinet_what_is_ddos}.

No caso concreto do ano de 2024, os ataques DDoS apresentaram um crescimento de ocorrências. Existem diversos fatores que contribuíram para este aumento, nomeadamente a realocação de uma quantidade substancial de infraestrutura critica para o ambiente \textit{online}, assim como a amplificação da cifra de dispositivos IoT que, por norma, possuem a ausência de proteções de segurança robustas, sendo a sua infeção com \textit{malware} e consequente integração numa BOTNET facilitada \cite{arnold_rise_2024}.

Este documento visa clarificar o leitor sobre os ataques DDoS, mediante a facultação de informação concisa e relevante sobre a temática. Nesta senda, cabe realçar que foi, também, realizado uma demonstração prática sobre a constituição de uma BOTNET, assim como a execução de um ataque DDoS, mediante a utilização desta, de modo a ilustrar, de forma mais tangível, o funcionamento destes ataques e a sua capacidade disruptiva.

Cumpre, ainda, esclarecer a estrutura do presente documento. Este relatório encontra-se organizado em \hl{7 capítulos}. Inicialmente é efetuada uma introdução ao contexto do problema, mais especificamente aos ataques DDoS. \hl{O segundo capítulo, procura proporcionar uma visão global sobre a temática, acrescentando detalhes cruciais à informação facultada no capítulo anterior. Por sua vez, o terceiro capítulo, aborda os mecanismos empregues pelos atacantes, aquando da constituição e execução de um ataque DDoS. No quarto capítulo é efetuada uma abordagem à demonstração prática elaborada. O quinto capítulo procura facultar informação relevante no âmbito das estratégias e desafios de mitigação de ataques DDoS. No caso concreto do sexto capítulo, cabe notar que este procura proporcionar uma perspetiva futura global sobre os ataques DDoS.} Por fim, o documento termina com uma breve conclusão sobre a temática.