\section{Perspetiva futura}
Os ataques de DDoS, atualmente, representam uma ameaça formidável ao panorama global da cibersegurança. Nesta senda, cabe notar que, o avanço da tecnologia proporciona novas possibilidades de atacantes constituirem novos ataques DDoS que empregam novos métodos, estratégias e vetores de ataque, procurando construir uma ofensiva progressivamente mais destrutiva. Assim, as organizações, de modo a garantir a proteção dos seus sistemas, devem possuir uma constante atenção ao mundo digital, mais especificamente aos avanços tecnológicos que possam ser utilizados para a constituição e concretização de novos ataques DDoS.


No âmbito de tendências emergentes relevantes ao contexto, importa realçar o crescimento de complexidade dos ataques DDoS, mais especificamente a utilização de múltiplos vetores de ataque. Deste modo, os ataques DDoS do futuro serão, possivelmente, muito mais sofisticados que os atuais e, consequentemente, muito mais destrutivos e dificil de mitigar. Além disso, o crescimento da quantidade de dispositivos IoT constituí uma ameaça substancial, já que estes possuem o potencial de aumentarem, significativamente, a quantidade de \textit{bots} em cada BOTNET, principalmente pois estes possuem uma ausência de defesas robustas que, consequentemente, facilitam a sua infeção por parte de um atacante. O crescimento da inteligência artificial, assim como dos algoritmos de \textit{machine learning} permitem a constituição de novas defesas contra ataques DDoS, explanado no capítulo 4. Contúdo, este aumento de preponderância destas tecnologias permite, também, a constituição de novos esquemas de ataques, proporcionando ataques dinâmicos com um elevado grau de adaptibilidade e, consequentemente, imensamente complexos e sofisticados. Mais se acrescenta que é possível perspetivar que os ataques DDoS possam ser, futuramente, utilizados como uma ferramenta de distração, de modo a viabilizar ataques informáticos mais sofisticados e complexos, que visam obter ilícitamente informações sensíveis e causar disrupções a infraestruturas críticas \cite{microminderscs_future_trends}.