\section{Conclusão}
O presente documento proporciona uma análise detalhada dos ataques DDoS. Deste modo, parece adequado afirmar que o objetivo de proporcionar informação suficiente e relevante sobre a temática foi atingido. Assim, cabe notar que são abordados os mecanismos operacionais, efetuadas categorizações e evidenciadas as ameaças que estes ataques representam no atual panorama global da cibersegurança.


Importa realçar que, mediante investigação ao conceito de BOTNETs, mais especificamente às suas características e arquiteturas, foi possível demonstrar informação pertinente sobre a temática, já que estas redes de computadores infetados são, frequentemente, empregues em ataques DDoS. Neste seguimento, a demonstração prática efetuada permitiu simular, em ambiente controlado, um ataque DDoS de pequena escala, sendo possível compreender, didaticamente, o \textit{modus operandi} destas mesmas redes, demonstrando, inclusivamente, a importância da escalabilidade nestas.

Cabe, ainda, notar que foram analisadas diversas complicações e estratégias presentes do processo de mitigação de um ataque DDoS, nomeadamente procedimentos de deteção, ferramentas defensivas, \textit{Cloud-based mitigation} e \textit{AI-Driven traffic analysis}. Mais se acrescenta que a evolução da tecnologia, permite aprimorar a sofisticação destas medidas defensivas, mas também possibilita o aumento da acessibilidade a ferramentas que viabilizam a realização de ataques DDoS, assim como o crescimento de dispositivos IoT ausentes de defesas robustas que podem, facilmente, constituir BOTNETs.


No âmbito de trabalho futuro importa salientar a necessidade de efetuar abordagens à temática que possibilitem o aperfeiçoamento do conhecimento sobre ataques DDoS que empregam múltiplos vetores de ataque, assim como ofensivas que utilizem inteligência artificial para aprimorar a sua capacidade destrutiva e dificultar a sua deteção.