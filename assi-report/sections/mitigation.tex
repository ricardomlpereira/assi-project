\section{Estratégias e desafios na mitigação}
\hl{TODO - Especificar que isto sao detacao atraves de sinais}

No âmbito da deteção de ataques DDoS, importa realçar que este processo envolve, essencialmente, o reconhecimento de sinais que indicam que o sistema encontra-se como alvo de ataque. Assim, revela-se imperativo notar estes mesmos sinais, a saber \cite{cybergc_defending_agaisnt_ddos}:
\begin{itemize}
    \item Aumento súbito e inesperado de tráfego de rede de uma localização específica ou de um endereço IP particular
    \item Desempenho fraco e irregular da rede, nomeadamente no âmbito dos tempos de carregamento dos \textit{websites} e da disponibilidade global dos serviços. Mais se acrescenta que um abrandamento considerável de um sistema é um sinal evidente de que este se encontra sob um ataque DDoS.
    \item Mensagens de erro de servidores, \textit{timeouts} e incapacidades em aceder a serviços e aplicações inexplicáveis. Assim, aquando destas circunstâncias, um atacante conseguiu comprometer a disponibilidade do sistema. Mais se acrescenta que, no caso concreto de um ataque DDoS de grau superior de seriedade, estas complicações não serão resolvidas mediante, exclusivamente, redução do tráfego de rede que chega a esta.
    \item Queixas, por parte dos funcionários, sobre uma conetividade de rede fraca. Mais se acrescenta que, este sinal, revela principal utilidade em circunstâncias onde a rede utilizada pelos funcionários é a mesma que a rede empregue pelos serviços, servidores e aplicações da organização.
    \item Desempenho transversalmente reduzido de dispositivos pertencente à mesma rede. Neste caso concreto, é possível perspetivar que o atacante comprometeu com sucesso a largura de banda da rede, minimizando a disponibilidade desta para os dispositivos que de si dependem.
    \item Notificações do \textit{Internet Service Provider},\textit{Cloud Service Provider}, assim como outros fornecedores de serviços empregues.
\end{itemize}

\hl{TODO - Abordar ferramentas em especifico}



Tópicos a abordar:
\begin{itemize}
    \item 2 páginas
    \item Difiuldades de deteção
    \item Escalabilidade das defesas
    \item Sofisticação dos ataques
    \item Assimetria entre recursos de ataque e defesa (...)
    \item como detetar um ataque DDoS
    \item Estrategias de ultima geração com grandes impactos
    \item AI-Driven trafic analysis; cloud-based mitigation; zero-trust network security; ect
    \item Estrategias mais tradicionais e ferramentas (maybe)
\end{itemize}