\section{Desafios e estratégias na mitigação}
A prevenção de ataques DDoS revela ser um processo de dificuldade significativa, dada a existência de múltiplos fatores que condicionam este processo. Neste seguimento, cabe notar a natureza distribuídas destes ataques complica este processo, dado que são, habitualmente, empregues BOTNETs constituídas por uma cifra imensa de dispositivos distintos, que não possibilitam um bloqueamento de tráfego eficiente, já que existe uma imensurável quantidade de fontes deste mesmo. Deste modo, importa, também, realçar o volume de tráfego associado a um ataque DDoS que, em ataques modernos, pode facilmente sobrecarregar a largura de banda de uma rede. Além disso, a variedade de ataques DDoS existente possuí a consequência de promover uma necessidade de adorar estratégias defensivas em camadas, mediante diversas técnicas, nomeadamente \textit{web application firewalls}, \textit{rate limiting}, DNS-\textit{Based traffic management}. Mais se acrescenta que os ataques DDoS tem apresentado um crescimento notório na sua sofisticação, mediante utilização de técnicas avançadas que permitem evadir processos de deteção e maximizar os impactos, assim como a utilização de múltiplos vetores de ataque consecutivamente e a crescente acessibilidade a serviços de \textit{DDoS-as-a-Service} que possibilitam, a utilizadores isentos de conhecimentos relevantes à execução de um ataque DDoS, realizar ataques, via realização, unicamente, de um pagamento \cite{perimeter81_ddos_motivations,radware_ddos_prevention,redhelix_rise_ddos_attacks}.

No âmbito da deteção de ataques DDoS, importa realçar que este processo envolve, essencialmente, o reconhecimento de sinais que indicam que o sistema encontra-se como alvo de ataque. Assim, revela-se imperativo notar estes mesmos sinais, a saber \cite{cybergc_defending_agaisnt_ddos}:
\begin{itemize}
    \item Aumento súbito e inesperado de tráfego de rede de uma localização específica ou de um endereço IP particular
    \item Desempenho fraco e irregular da rede, nomeadamente no âmbito dos tempos de carregamento dos \textit{websites} e da disponibilidade global dos serviços. Mais se acrescenta que um abrandamento considerável de um sistema é um sinal evidente de que este se encontra sob um ataque DDoS.
    \item Mensagens de erro de servidores, \textit{timeouts} e incapacidades em aceder a serviços e aplicações inexplicáveis. Assim, aquando destas circunstâncias, um atacante conseguiu comprometer a disponibilidade do sistema. Mais se acrescenta que, no caso concreto de um ataque DDoS de grau superior de seriedade, estas complicações não serão resolvidas mediante, exclusivamente, redução do tráfego de rede que chega a esta.
    \item Queixas, por parte dos funcionários, sobre uma conetividade de rede fraca. Mais se acrescenta que, este sinal, revela principal utilidade em circunstâncias onde a rede utilizada pelos funcionários é a mesma que a rede empregue pelos serviços, servidores e aplicações da organização.
    \item Desempenho transversalmente reduzido de dispositivos pertencente à mesma rede. Neste caso concreto, é possível perspetivar que o atacante comprometeu com sucesso a largura de banda da rede, minimizando a disponibilidade desta para os dispositivos que de si dependem.
    \item Notificações do \textit{Internet Service Provider},\textit{Cloud Service Provider}, assim como outros fornecedores de serviços empregues.
\end{itemize}

Nesta senda, cabe, também, notar que existem métodos automatizados que permitem a deteção de ataques DDoS, nomeadamente ferramentas como \textit{Intrusion Detection Systems} e \textit{Security Information and Event Management}, que proporcionam um alerta célere e eficiente dos administradores, numa circunstância em que algum sistema, abrangido por estes, seja alvo de um ataque \cite{suntera_rise_ddos}.

No caso concreto de proteção e mitigação, releva realçar que são adotadas diversas técnicas e estratégias, de modo a redirecionar, para um \textit{scrubbing center} ou utilizando \textit{load balancers}, o tráfego resultante de um ataque DDoS, de forma célere e eficiente. Além disso, as organizações, atualmente, adotam, também, diversas tecnologias que permitem identificar e intercetar tráfego malicioso, a saber \cite{ibm_what_is_ddos}:
\begin{itemize}
    \item \textit{Web application firewalls}: quando utilizadas em conjunto com outras defesas periféricas, permitem proteger a rede, assim como as aplicações de atividade maliciosa, mediante análise a cada pacote no tráfego de rede, de modo a identificar e bloquear tráfego malicioso.
    \item Content Delivery Networks: permitem agilizar o acesso dos utilizadores a recursos \textit{online}, via a utilização de uma rede de servidor distribuídos, sendo que, no caso concreto em que esta é impactada por um ataque DDoS, é possível alterar o trajeto do tráfego legítimo para outros servidores que possuam recursos disponíveis.
    \item \textit{Security Information and Event Management}: proporcionam diversas funções de deteção de ataques DDoS antecipadamente, mediante a análise de eventos de segurança, assim como a monitorização de tráfego de rede, alertando os administradores, aquando da deteção de atividade maliciosa.
\end{itemize}

Nesta senda, cabe, também, notar a existência de tecnologias de proteção e mitigação de ataques DDoS de última geração, a saber: \textit{Cloud-based mitigation}; \textit{AI-Driven traffic analysis}.


A primeira, proporciona proteções aos sistemas, via utilização de infraestruturas em nuvem, de modo a absorver, filtrar e mitigar tráfego malicioso, previamente à chegada deste ao sistema. Neste seguimento, esta tecnologia possuí no seu cerne 4 etapas distintas, de modo a combater o ataque, a saber: Deteção, procurando identificar tráfego malicioso; Resposta, visando abandonar tráfego malicioso; Encaminhamento, procurando proporcionar um redirecionamento eficiente de tráfego; Adaptação, visando ajustar as defesas do sistema \cite{cloudflare_cloud_based_protection}.


Por sua vez, a segunda, possibilita a identificação de tráfego malicioso de forma célere e eficiente, via utilização de algoritmos de \textit{Machine Learning} que analisam o tráfego de uma rede continuamente, de modo a identificar anomalias e a bloquear ameaças. Nesta senda, cabe notar que, a utilização destes algoritmos possibilita uma deteção de ameaças dinâmica, já que é possível alterar as regras de deteção em tempo real, permitindo ajustar estas a cada circunstância específica. Assim, esta adaptabilidade faculta proteções consideráveis no âmbito de ataques DDoS que empregam múltiplos vetores de ataque. Mais se acrescenta que o processamento, por parte dos algoritmos, de um volume progressivamente crescente de dados, permite que estes efetuem um aprimoramento significativo das suas capacidades, sendo que estas novas aptidões revelam-se cruciais na manutenção do desempenho e disponibilidade dos sistemas, em circunstâncias onde são requisitadas respostas imediatas e precisas \cite{edgenext_ai_ddos_protection}.