\section{Panorama global}
Os ataques DDoS são uma ameaça informática frequentemente empregue, de modo a comprometer, principalmente, a disponibilidade de serviços, servidores, aplicações e redes. Neste seguimento, cabe notar que, estes ataques, possuem diversas características que os tornam únicos no âmbito dos ataques informáticos, mas, também, na categoria das ofensivas de DoS, nomeadamente na metodologia e ferramentas empregues para a sua concretização.
\subsection{Conceito de BOTNET}
Uma BOTNET é, na sua essência, uma rede de dispositivos eletrónicos sequestrados, mediante a utilização de diversas categorias de \textit{malware}, por um atacante, de modo a possibilitar a realização de um ataque de negação de serviço em ampla escala, mas também de ataques de \textit{phishing} e de \textit{brute force}. Nesta senda, os dispositivos que compõem uma BOTNET são, principalmente, computadores, servidores, routers, câmaras de vigilância, e diversos outros aparelhos, nomeadamente dispositivos IoT. Assim, aquando da sua infeção, o atacante consegue, remotamente, controlar estes aparelhos, frequentemente sem o conhecimento e autorização dos seus proprietários. Estas redes revelam um elevado grau de distributividade e tolerância a falhas, multifuncionalidade, assim como persistência e inteligencia \cite{kaspersky_botnets_2017, paloaltonetwork_botnets}.

No âmbito do funcionamento das BOTNETs, importa realçar que estas são construídas com base na sua futura expansão e automatização, assim como de modo a possibilitar amplificação da potencia e da velocidade de ataques de larga escala. Deste modo, um dispositivo pertencente a esta rede, denominado por \textit{zombie} ou \textit{bot}, é controlado, via instruções remotas, por um dispositivo central, sobre o total dominio do atacante, designado por \textit{bot herder}. Nesta senda, cabe, também, notar que estas redes possuem um potencial imensamente destrutivo, já que são frequentemente vendidas ou alugadas, pelo seu criador, a indivíduos ou entidades, de modo a concretizar uma ou mais das diversas motivações inframenciondas \cite{kaspersky_botnets_2017, paloaltonetwork_botnets}.

Neste seguimento, o atacante, via o dispositivo \textit{bot herder}, emprega a estratégia de \textit{Command-and-control} que permite, mediante a utilização de um servidor central, faculta a transmissão de instruções, remotamente, aos \textit{bots} pertencentes à BOTNET, seguindo, também, modelos de comunicação centralizados, nomeadamente o paradigma \textit{client-server}, ou descentralizados, principalmente o tipo \textit{peer-to-peer}.

No caso concreto do modelo centralizado, cabe notar que este é organizado hierárquicamente, ou seja, existe um servidor central que possuí a função de transmitir todos os comandos aos \textit{bots} podendo, também, existir intermediários entre a cimeira e a base da hierarquia denominados por \textit{sub-herders} ou \textit{proxies}. Por sua vez, os modelos descentralizados, atribuem a responsabilidade de transmissão de instruções a todos os dispositivos da BOTNET, inclusivamente os próprios \textit{bots}, já que, basta o \textit{bot herder} conseguir enviar os comandos para um dos \textit{bots} para que estas sejam transmitidas a todos os elementos da rede em questão. Nesta senda, importa, ainda, realçar que o modelo descentralizado possuí, atualmente, uma maior ocorrência, dado que este permite uma ofuscação superior do atacante, assim como do \textit{bot herder} \cite{kaspersky_botnets_2017}.


Releva, ainda, notar que, posteriormente à infeção, o \textit{bot} permite ao atacante, mais especificamente, acesso a operações \textit{admin-level}, nomeadamente a leitura e escrita de dados de sistema, coleta de informação pessoal do proprietário, envio de dados, monitorizamento das atividades do utilizador, procura por vulnerabilidades e a instalação e execução de qualquer aplicação \cite{kaspersky_botnets_2017}.

Assim, a construção de uma BOTNET pode ser organizada em 3 fases distintas, a saber:
\begin{itemize}
    \item Preparação e Exposição: o atacante, incialmente, procura por uma vulnerabilidade num dos tipos de dispositivos supramencionados, visando descobrir uma falha que possibilite uma infeção sem rastro deste mesmo dispositivo \cite{kaspersky_botnets_2017}.
    \item Infeção: o utilizador, ao executar uma ação, compromete o seu próprio dispositivo com o \textit{malware} do atacante, tornado este num \textit{bot}. Nesta senda, cabe notar que existem diversas formas de concretizar a infeção, nomeadamente via estratégias de \textit{social engineering}, mas também mediante outras técnicas mais agressivas, principalmente o caso dos \textit{drive-by downloads} \cite{kaspersky_botnets_2017}.
    \item Ativação: o atacante possuí controlo total do \textit{bot}, procedendo à sua adição à rede de BOTNET e, posteriormente, à sua utilização num ataque de negação de serviço, \textit{phishing} ou \textit{brute force} \cite{kaspersky_botnets_2017}.
\end{itemize}

No âmbito de motivações, releva realçar que estas redes são frequentemente empregues em ações de furtos de fundos monetários e de informação pessoal, assim como outros tipos de esquemas, revelando curialidade à técnica de extorsão e a diversas práticas de roubo, assim como na constituição de esquemas. Além disso, são, também, utilizadas na sabotagem de serviços, já que estas possuem a capacidade de concretizar ataques DDoS poderosos \cite{kaspersky_botnets_2017}.

\subsection{Categorias de ataques DDoS}
A categorização das distinções entre ataques DDoS permite aprimorar a compreensão que indivíduo possuí destes. Nesta seda, cabe notar que existem 3 categorias principais de ataques de DDoS, a saber: \textit{Volumetric DDoS Attacks};\textit{Protocol Attacks}; \textit{Application Attacks} \cite{esecurityplanet_types_of_ddos_attacks,connectwise_types_of_ddos_attacks}.

As 3 categorias supramencionados facilitam a caracterização da grande parte de ataques DDoS. Contúdo, nem todos os ataques de negação de serviço em ampla escala podem ser balizados nas categorias definidas, principalmente ataques \textit{Advanced Persistent} DoS, ataques que emprega múltiplos vetores e ataques DDoS \textit{Zero-Day}. Assim, os atacantes empregam, frequentemente, uma combinação de técnicas e métodos distintos, de modo a fortalecer a robustez dos seus ataques, viabilizando a continuação do seu impacto, já que estes se tornam considerávelmente mais complicados de detetar e mitigar \cite{esecurityplanet_types_of_ddos_attacks,connectwise_types_of_ddos_attacks}.

\subsubsection{\textit{Volumetric DDoS Attacks}}

No caso concreto dos ataques DDoS volumétricos, cabe notar que estes procuram sobrecarregar a capacidade de recursos de um dispositivo alvo, mediante utilização de um elemento curial a este processo, nomeadamente pedidos, tráfego e chamadas a, respetivamente, servidores, redes e bases de dados. Nesta senda, um ataque volumétrico de negação de serviço distribuída permite saturar a largura de banda do alvo, sendo a magnitude do ataque medida em \textit{bits} transmitidos por segundo. Assim, releva realçar que os \textit{Volumetric DDoS Attacks} incluem os diversos ataques de \textit{flooding} inframencionados, nomeadamente \textit{User Datagram Protocol} (UDP) \textit{Flooding}, CharGEN \textit{Flooding} e Internet Control Message Protocol (ICMP) \textit{Flooding}, assim como ataques que empregam aplicações indevidamente utilizadas \cite{esecurityplanet_types_of_ddos_attacks}.

Neste seguimento, os ataques de UDP \textit{Flooding} procuram enviar uma cifra desproporcionada de pacotes UDP ao alvo, de modo a sobrecarregar a sua capacidade de processamento destes mesmos, assim como exaustar a sua largura de banda. Assim, cabe notar que este ataque procura explorar a natureza do protocolo UDP, mais especificamente, a sua característica de não ser estabelecida uma conexão entre o emissor e o recetor, sendo apenas efetuada uma tentativa de envio do pacote, sem qualquer tentativa de obter uma resposta. Deste modo, os atacantes visam servidores presentes na Internet ou numa rede, via os seus endereços \textit{Internet Protocol} (IP) e as portas associadas ao protocolo UDP \cite{esecurityplanet_types_of_ddos_attacks}. Importa, ainda, realçar que este ataque incluí as variantes UDP \textit{Fragmentation Flooding} e \textit{Specific} UDP \textit{Amplification Attacks}. A primeira variante, procura enviar pacotes UDP fragmentados de superior dimensão, visando que o alvo, ao receber estes e aquando do consequente processo de junção de fragmentos, fique sobrecarregado. Por sua vez, a segunda variante, procura enviar um único pedido UDP verdadeiro,utilizando o endereço IP do alvo, para diversos servidores, de modo a que estes respondam ao alvo e, consequentemente, promovam uma sobrecarga neste. Nesta senda, cabe notar que são empregues diversos protocolos que utilizam o modelo UDP, de modo a concretizar este ataque específico, a saber: Network Time Protocol; Simple Network Management Protocol; e Simple Service Discovery Protocol \cite{esecurityplanet_types_of_ddos_attacks}.

Por sua vez, o CharGEN \textit{Flooding} procura explorar o funcionamento do protocol CharGEN, mais especificamente, o seu procedimento de responder a pedidos \textit{Transmission Control Protocol} (TCP) ou UDP via porta 19 com, respetivamente, caracteres arbitrariamente gerados e números aleatórios. Assim, um atacante pode efetuar o \textit{spoofing} do endereço IP do seu alvo, enviando, posteriormente, uma quantidade substancial de pedidos TCP ou UDP a dispositivos que empregam o protocolo CharGEN, nomeadamente impressoras e fotocopiadores que, ao responderem aos a estes pedidos da forma supramencionada, vão promover uma quantidade considerável de tráfego na porta 19 do sistema alvo. Deste modo, no caso concreto da \textit{firewall} do sistema alvo não bloquear esta mesma porta, o servidor pode ficar sobrecarregado com o processo de análise e posterior resposta a todos os pedidos que recebeu \cite{esecurityplanet_types_of_ddos_attacks,connectwise_types_of_ddos_attacks}.

O protocol ICMP consiste na comunicação entre dispositivos de rede, mediante o envio de mensagens de erro específicas e de comandos de informação operacional, nomeadamente \textit{Timestamp}, \textit{Time Exceeded error}, \textit{Echo Request} e \textit{Echo Reply}. Assim, importa notar que estas 2 últimas mensagens podem ser empregues de forma conjunta, de modo a constituir o comando \textit{ping}. Deste modo, cabe realçar a possibilidade de caracterizar o ICMP \textit{Flooding} de 2 formas distintas, a saber: \textit{Ping Flooding} e \textit{Fragmentation Flooding}. No primeiro caso, os atacantes utilizam uma quantidade substancial de dispositivos para enviar pacotes \textit{spoofed} ICMP \textit{Ping} a diversos servidores. Assim, estes servidores, de modo a cumprirem os requisitos estabelecidos no protocolo ICMP, necessitam de responder aos pedidos, mediante o envio de um pacote resposta para o sistema alvo, já que os pacotes ICMP \textit{Ping} possuem o endereço IP deste mesmo como emissor, promovendo uma sobrecarga deste mesmo, dado que irá receber num curto espaço temporal uma quantidade enorme de pacotes. Por sua vez, no segundo caso, os atacantes visam substituir cada pacote ICMP com o seu respetivo comando por diversos pacotes, de modo a fragmentar este mesmo. Assim, o sistema alvo, ao receber estes pacotes, vai proceder à sua reconstrução, de modo a obter o comando no seu estado original e, consequentemente, vai exaustar recursos ao tentar estabelecer conexões entre fragmentos intencionalmente não relacionados \cite{esecurityplanet_types_of_ddos_attacks}.

No caso concreto dos ataques que procuram explorar aplicações indevidamente utilizadas, importa realçar que, os atacantes, visam comprometer aplicações de alto tráfego em servidores legítimos, nomeadamente servidores \textit{peer-to-peer}, de modo a que o tráfego destas seja redirecionado para um sistema alvo, permitindo, assim, que o atacante saia do sistema e que este funcione autónomamente. Nesta senda, importa realçar que as aplicações comprometidas vão procurar estabelecer conexões válidas com o sistema alvo, promovendo a sobrecargar deste. Mais se acrescenta que, a grande parte das ferramentas de proteção frequentemente utilizadas vão permitir estas tentativas de conexão, já que os pacotes enviados pelas aplicações comprometidas vão ser corretamente formatados e compostos \cite{esecurityplanet_types_of_ddos_attacks}.

Importa, ainda, notar que no ano de 2020 o serviço \textit{Amazon Web Services} foi alvo de um \textit{Volumetric DDoS Attack} de 2.3 \textit{Terabits} por segundo de largura de banda, representado este um dos mais recentes e impactantes ataques da sua categoria. Neste seguimento, o ataque em questão foi concretizado via exploração do protocolo \textit{Connection-less Lightweight Directory Access Protocol}, de modo a inundar serviço em questão com um volume significativo de tráfego irrelevante. Mais se acrescenta que foram necessários diversos dias para que a equipa encarregue de solucionar este constragimento fosse capaz de mitigar por completo o ataque \cite{connectwise_types_of_ddos_attacks,aws_ddos_2020}.

\subsubsection{\textit{Protocol Attacks}}
TODO


\subsubsection{\textit{Protocol Attacks}}
TODO


\subsection{\textit{Mecanismos de Deteção}}
TODO


\subsection{\textit{Ferramentas e plataformas utilizadas}}
TODO


\subsection{\textit{Motivações e Impactos}}
TODO