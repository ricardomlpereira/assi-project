\section{Panorama global}
Os ataques DDoS são uma ameaça informática frequentemente empregue, de modo a comprometer, principalmente, a disponibilidade de serviços, servidores, aplicações e redes. Neste seguimento, cabe notar que, estes ataques, possuem diversas características que os tornam únicos no âmbito dos ataques informáticos, mas, também, na categoria das ofensivas de DoS, nomeadamente na metodologia e ferramentas empregues para a sua concretização.
\subsection{Conceito de BOTNET}
Uma BOTNET é, na sua essência, uma rede de dispositivos eletrónicos sequestrados, mediante a utilização de diversas categorias de \textit{malware}, por um atacante, de modo a possibilitar a realização de um ataque de negação de serviço em ampla escala, mas também de ataques de \textit{phishing} e de \textit{brute force}. Nesta senda, os dispositivos que compõem uma BOTNET são, principalmente, computadores, servidores, routers, câmaras de vigilância, e diversos outros aparelhos, nomeadamente dispositivos IoT. Assim, aquando da sua infeção, o atacante consegue, remotamente, controlar estes aparelhos, frequentemente sem o conhecimento e autorização dos seus proprietários. Estas redes revelam um elevado grau de distributividade e tolerância a falhas, multifuncionalidade, assim como persistência e inteligencia \cite{kaspersky_botnets_2017, paloaltonetwork_botnets}.

No âmbito do funcionamento das BOTNETs, importa realçar que estas são construídas com base na sua futura expansão e automatização, assim como de modo a possibilitar amplificação da potencia e da velocidade de ataques de larga escala. Deste modo, um dispositivo pertencente a esta rede, denominado por \textit{zombie} ou \textit{bot}, é controlado, via instruções remotas, por um dispositivo central, sobre o total dominio do atacante, designado por \textit{bot herder}. Nesta senda, cabe, também, notar que estas redes possuem um potencial imensamente destrutivo, já que são frequentemente vendidas ou alugadas, pelo seu criador, a indivíduos ou entidades, de modo a concretizar uma ou mais das diversas motivações inframenciondas \cite{kaspersky_botnets_2017, paloaltonetwork_botnets}.

Neste seguimento, o atacante, via o dispositivo \textit{bot herder}, emprega a estratégia de \textit{Command-and-control} que permite, mediante a utilização de um servidor central, faculta a transmissão de instruções, remotamente, aos \textit{bots} pertencentes à BOTNET, seguindo, também, modelos de comunicação centralizados, nomeadamente o paradigma \textit{client-server}, ou descentralizados, principalmente o tipo \textit{peer-to-peer}.

No caso concreto do modelo centralizado, cabe notar que este é organizado hierárquicamente, ou seja, existe um servidor central que possuí a função de transmitir todos os comandos aos \textit{bots} podendo, também, existir intermediários entre a cimeira e a base da hierarquia denominados por \textit{sub-herders} ou \textit{proxies}. Por sua vez, os modelos descentralizados, atribuem a responsabilidade de transmissão de instruções a todos os dispositivos da BOTNET, inclusivamente os próprios \textit{bots}, já que, basta o \textit{bot herder} conseguir enviar os comandos para um dos \textit{bots} para que estas sejam transmitidas a todos os elementos da rede em questão. Nesta senda, importa, ainda, realçar que o modelo descentralizado possuí, atualmente, uma maior ocorrência, dado que este permite uma ofuscação superior do atacante, assim como do \textit{bot herder} \cite{kaspersky_botnets_2017}.


Releva, ainda, notar que, posteriormente à infeção, o \textit{bot} permite ao atacante, mais especificamente, acesso a operações \textit{admin-level}, nomeadamente a leitura e escrita de dados de sistema, coleta de informação pessoal do proprietário, envio de dados, monitorizamento das atividades do utilizador, procura por vulnerabilidades e a instalação e execução de qualquer aplicação \cite{kaspersky_botnets_2017}.

Assim, a construção de uma BOTNET pode ser organizada em 3 fases distintas, a saber \cite{kaspersky_botnets_2017}:
\begin{itemize}
    \item Preparação e Exposição: o atacante, incialmente, procura por uma vulnerabilidade num dos tipos de dispositivos supramencionados, visando descobrir uma falha que possibilite uma infeção sem rastro deste mesmo dispositivo.
    \item Infeção: o utilizador, ao executar uma ação, compromete o seu próprio dispositivo com o \textit{malware} do atacante, tornado este num \textit{bot}. Nesta senda, cabe notar que existem diversas formas de concretizar a infeção, nomeadamente via estratégias de \textit{social engineering}, mas também mediante outras técnicas mais agressivas, principalmente o caso dos \textit{drive-by downloads}.
    \item Ativação: o atacante possuí controlo total do \textit{bot}, procedendo à sua adição à rede de BOTNET e, posteriormente, à sua utilização num ataque de negação de serviço, \textit{phishing} ou \textit{brute force}.
\end{itemize}

No âmbito de motivações, releva realçar que estas redes são frequentemente empregues em ações de furtos de fundos monetários e de informação pessoal, assim como outros tipos de esquemas, revelando curialidade à técnica de extorsão e a diversas práticas de roubo, assim como na constituição de esquemas. Além disso, são, também, utilizadas na sabotagem de serviços, já que estas possuem a capacidade de concretizar ataques DDoS poderosos \cite{kaspersky_botnets_2017}.

\subsection{Categorias de ataques DDoS}
A categorização das distinções entre ataques DDoS permite aprimorar a compreensão que indivíduo possuí destes. Nesta seda, cabe notar que existem 3 categorias principais de ataques de DDoS, a saber: \textit{Volumetric DDoS Attacks};\textit{Protocol Attacks}; \textit{Application Attacks} \cite{esecurityplanet_types_of_ddos_attacks,connectwise_types_of_ddos_attacks}.

As 3 categorias supramencionados facilitam a caracterização da grande parte de ataques DDoS. Contúdo, nem todos os ataques de negação de serviço em ampla escala podem ser balizados nas categorias definidas, principalmente ataques \textit{Advanced Persistent} DoS, ataques que emprega múltiplos vetores e ataques DDoS \textit{Zero-Day}. Assim, os atacantes empregam, frequentemente, uma combinação de técnicas e métodos distintos, de modo a fortalecer a robustez dos seus ataques, viabilizando a continuação do seu impacto, já que estes se tornam considerávelmente mais complicados de detetar e mitigar \cite{esecurityplanet_types_of_ddos_attacks,connectwise_types_of_ddos_attacks}.

\subsubsection{\textit{Volumetric DDoS Attacks}}

No caso concreto dos ataques DDoS volumétricos, cabe notar que estes procuram sobrecarregar a capacidade de recursos de um dispositivo alvo, mediante utilização de um elemento curial a este processo, nomeadamente pedidos, tráfego e chamadas a, respetivamente, servidores, redes e bases de dados. Nesta senda, um ataque volumétrico de negação de serviço distribuída permite saturar a largura de banda do alvo, sendo a magnitude do ataque medida em \textit{bits} transmitidos por segundo. Assim, releva realçar que os \textit{Volumetric DDoS Attacks} incluem os diversos ataques de \textit{flooding} inframencionados, nomeadamente \textit{User Datagram Protocol} (UDP) \textit{Flooding}, CharGEN \textit{Flooding} e Internet Control Message Protocol (ICMP) \textit{Flooding}, assim como ataques que empregam aplicações indevidamente utilizadas \cite{esecurityplanet_types_of_ddos_attacks}.

Neste seguimento, os ataques de UDP \textit{Flooding} procuram enviar uma cifra desproporcionada de pacotes UDP ao alvo, de modo a sobrecarregar a sua capacidade de processamento destes mesmos, assim como exaustar a sua largura de banda. Assim, cabe notar que este ataque procura explorar a natureza do protocolo UDP, mais especificamente, a sua característica de não ser estabelecida uma conexão entre o emissor e o recetor, sendo apenas efetuada uma tentativa de envio do pacote, sem qualquer tentativa de obter uma resposta. Deste modo, os atacantes visam servidores presentes na Internet ou numa rede, via os seus endereços \textit{Internet Protocol} (IP) e as portas associadas ao protocolo UDP \cite{esecurityplanet_types_of_ddos_attacks}. Importa, ainda, realçar que este ataque incluí as variantes UDP \textit{Fragmentation Flooding} e \textit{Specific} UDP \textit{Amplification Attacks}. A primeira variante, procura enviar pacotes UDP fragmentados de superior dimensão, visando que o alvo, ao receber estes e aquando do consequente processo de junção de fragmentos, fique sobrecarregado. Por sua vez, a segunda variante, procura enviar um único pedido UDP verdadeiro,utilizando o endereço IP do alvo, para diversos servidores, de modo a que estes respondam ao alvo e, consequentemente, promovam uma sobrecarga neste. Nesta senda, cabe notar que são empregues diversos protocolos que utilizam o modelo UDP, de modo a concretizar este ataque específico, a saber: Network Time Protocol; Simple Network Management Protocol; e Simple Service Discovery Protocol \cite{esecurityplanet_types_of_ddos_attacks}.

Por sua vez, o CharGEN \textit{Flooding} procura explorar o funcionamento do protocol CharGEN, mais especificamente, o seu procedimento de responder a pedidos \textit{Transmission Control Protocol} (TCP) ou UDP via porta 19 com, respetivamente, caracteres arbitrariamente gerados e números aleatórios. Assim, um atacante pode efetuar o \textit{spoofing} do endereço IP do seu alvo, enviando, posteriormente, uma quantidade substancial de pedidos TCP ou UDP a dispositivos que empregam o protocolo CharGEN, nomeadamente impressoras e fotocopiadores que, ao responderem aos a estes pedidos da forma supramencionada, vão promover uma quantidade considerável de tráfego na porta 19 do sistema alvo. Deste modo, no caso concreto da \textit{firewall} do sistema alvo não bloquear esta mesma porta, o servidor pode ficar sobrecarregado com o processo de análise e posterior resposta a todos os pedidos que recebeu \cite{esecurityplanet_types_of_ddos_attacks,connectwise_types_of_ddos_attacks}.

O protocol ICMP consiste na comunicação entre dispositivos de rede, mediante o envio de mensagens de erro específicas e de comandos de informação operacional, nomeadamente \textit{Timestamp}, \textit{Time Exceeded error}, \textit{Echo Request} e \textit{Echo Reply}. Assim, importa notar que estas 2 últimas mensagens podem ser empregues de forma conjunta, de modo a constituir o comando \textit{ping}. Deste modo, cabe realçar a possibilidade de caracterizar o ICMP \textit{Flooding} de 2 formas distintas, a saber: \textit{Ping Flooding} e \textit{Fragmentation Flooding}. No primeiro caso, os atacantes utilizam uma quantidade substancial de dispositivos para enviar pacotes \textit{spoofed} ICMP \textit{Ping} a diversos servidores. Assim, estes servidores, de modo a cumprirem os requisitos estabelecidos no protocolo ICMP, necessitam de responder aos pedidos, mediante o envio de um pacote resposta para o sistema alvo, já que os pacotes ICMP \textit{Ping} possuem o endereço IP deste mesmo como emissor, promovendo uma sobrecarga deste mesmo, dado que irá receber num curto espaço temporal uma quantidade enorme de pacotes. Por sua vez, no segundo caso, os atacantes visam substituir cada pacote ICMP com o seu respetivo comando por diversos pacotes, de modo a fragmentar este mesmo. Assim, o sistema alvo, ao receber estes pacotes, vai proceder à sua reconstrução, de modo a obter o comando no seu estado original e, consequentemente, vai exaustar recursos ao tentar estabelecer conexões entre fragmentos intencionalmente não relacionados \cite{esecurityplanet_types_of_ddos_attacks}.

No caso concreto dos ataques que procuram explorar aplicações indevidamente utilizadas, importa realçar que, os atacantes, visam comprometer aplicações de alto tráfego em servidores legítimos, nomeadamente servidores \textit{peer-to-peer}, de modo a que o tráfego destas seja redirecionado para um sistema alvo, permitindo, assim, que o atacante saia do sistema e que este funcione autónomamente. Nesta senda, importa realçar que as aplicações comprometidas vão procurar estabelecer conexões válidas com o sistema alvo, promovendo a sobrecargar deste. Mais se acrescenta que, a grande parte das ferramentas de proteção frequentemente utilizadas vão permitir estas tentativas de conexão, já que os pacotes enviados pelas aplicações comprometidas vão ser corretamente formatados e compostos \cite{esecurityplanet_types_of_ddos_attacks}.

Importa, ainda, notar que no ano de 2020 o serviço \textit{Amazon Web Services} foi alvo de um \textit{Volumetric DDoS Attack} de 2.3 \textit{Terabits} por segundo de largura de banda, representado este um dos mais recentes e impactantes ataques da sua categoria. Neste seguimento, o ataque em questão foi concretizado via exploração do protocolo \textit{Connection-less Lightweight Directory Access Protocol}, de modo a inundar serviço em questão com um volume significativo de tráfego irrelevante. Mais se acrescenta que foram necessários diversos dias para que a equipa encarregue de solucionar este constragimento fosse capaz de mitigar por completo o ataque \cite{connectwise_types_of_ddos_attacks,aws_ddos_2020}.

\subsubsection{\textit{Protocol DDoS Attacks}}
Os \textit{Protocol Attacks} procuram explorar protocolos, de modo a sobrecarregar um recurso público específico, nomeadamente servidores mas, também, \textit{firewalls} ou \textit{load balancers} constrastando, assim, como os \textit{Volumetric DDoS Attacks}, dado que não é utilizada uma quantidade substancial de uma matéria, principalmente \textit{requests}, para concretizar o ataque de negação de servíço distribuída, mas sim apenas uma falha. Mais se acrescenta que a magnitude dos \textit{Protocol Attacks} é medida mediante cálculo dos pacotes por segundo que este envia \cite{esecurityplanet_types_of_ddos_attacks}.


Neste seguimento, cabe notar os ataques IP \textit{Null} onde os atacantes procuram definir o \textit{header} de um pacote IPv4 como nulo e sem instruções específicas no âmbito da descarta do pacote, de modo a promover que o servidor destinatário deste consuma recursos substanciais numa tentativa fútil de determinar como proceder com o processo de comunicação, mais especificamente a entrega do pacote \cite{esecurityplanet_types_of_ddos_attacks}.

Importa, também, realçar os ataques de TCP \textit{Flooding} que procuram explorar diretamente as características do protocolo TCP. Assim, estes ataques visam exaustar os recursos do sistema alvo, utilizando a estrutura do próprio protocolo, assim como via utilização de técnicas de \textit{spoofing} e pacotes mal formados. Nesta senda, cabe notar que estas práticas permitem, a um atacante, constituir uma disrupção substancial no âmbito da disponibilidade do seu alvo, sendo, consequentemente, relevante realçar as diversas categorias distintas de ataques de TCP \textit{Flooding} existentes, a saber \cite{esecurityplanet_types_of_ddos_attacks,connectwise_types_of_ddos_attacks}:
\begin{itemize}
    \item SYN \textit{Flooding}
    \item SYN-ACK \textit{Flooding}
    \item ACK \textit{Flooding}
    \item ACK \textit{Fragmentation Flooding}
    \item RST/FIN \textit{Flooding}
    \item \textit{Multiple} ACK \textit{Spoofed Session Flood}
    \item \textit{Multiple} SYN-ACK \textit{Spoofed Session Flood}
    \item \textit{Synonymous} IP \textit{Attack}
\end{itemize}


Além dos ataques que procuram explorar diretamente o protocolo TCP, importa notar outras abordagens que empregam diversos protocolo e técnicas, de modo a concretizar ataques de negação de serviço distribuídos. Neste seguimento, cabe notar que, estes ataques, utilizam, frequentemente, vulnerabilidades inerentes de protocolos habitualmente empregues por inúmeros serviços, aplicações e servidores na Internet. Além disso são, também, executadas estratégias de \textit{flooding}, de modo a aumentar a quantidade de tráfego enviado e, consequentemente, a cifra de recursos consumidos pelo sistema alvo. Mais se acrescenta que estes ataques, apesar de possuírem diversas implementações distintas, partilham o mesmo intuito de comprometer a disponibilidade do seu alvo, assim como possuem a capacidade de serem empregues em conjunto com outros vetores de ataque, promovendo uma dificuldade considerável no âmbito da sua mitigação. Deste modo, cumpre realçar as diversas categorias de ataques abrangidos pelas características supramencionadas, a saber \cite{esecurityplanet_types_of_ddos_attacks,connectwise_types_of_ddos_attacks}:
\begin{itemize}
    \item \textit{Session Attacks}
    \item Slowloris
    \item \textit{Ping of Death}
    \item \textit{Smurf Attack}
    \item \textit{Fraggle Attack}
    \item \textit{Low Orbit Ion Cannon}
    \item \textit{High Orbit Ion Cannon}
\end{itemize}

\subsubsection{\textit{Application DDoS Attacks}}
No âmbito dos DDoS \textit{Application Attacks} importa realçar que estes procuram explorar vulnerabilidades em aplicações, de modo a causar falhas na própria aplicação. Deste modo, cabe notar que, ao contrário de outras categorias de ataques DDoS, mais específicamente aqueles que visam comprometer a disponibilidade de um dispositivo ou infraestrutura, os ataques de aplicações concentram-se em comprometer a camada 7 do modelo \textit{Open Systems Interconnection} (OSI). Contudo, estes ataques podem, também, resultar em processadores sobrecarregados, assim como memórias exaustas, afetando, inclusivamente, o servidor e outras aplicações. Assim, releva realçar que a magnitude de um \textit{Application Attack} é medida em pedidos por segundo. Mais se acrescenta que os principais ataques DDoS de aplicações incluem \textit{Hypertext Transfer Protocol} (HTTP) \textit{Flood Attacks} e ReDoS \cite{esecurityplanet_types_of_ddos_attacks,connectwise_types_of_ddos_attacks}.


Nesta senda, cabe notar que os ataques de HTTP \textit{Flood} procuram explorar os comandos HTTP, de modo a sobrecarregar \textit{websites}, os servidores que hospedam estes, assim como a largura de banda emprega para os alcançar, mediante utilização de uma BOTNET. Assim, os \textit{bots} utilizados nestes ataques permitem o envio de múltiplos pedidos sequencialmente, de modo a que cada \textit{bot} constituinte da rede permita aumentar exponencialmente o tráfego enviado para o \textit{website} alvo. Deste modo, importa realçar os diversos tipos de ataques de HTTP \textit{Flood} existentes, a saber \cite{esecurityplanet_types_of_ddos_attacks}:
\begin{itemize}
    \item Ataques de GET: o atacante procura enviar um volume substancial de pedidos concurrentes GET de ficheiros de dimensão considerável, nomeadamente vídeos e ficheiros \textit{Portable Document Format} (PDF).
    \item Ataques de POST: o atacante visa enviar um volume considerável de pedidos concurrentes POST de ficheiros de dimensão notáveel, nomeadamente vídeos e ficheiros PDF.
    \item \textit{Long-and-Slow} POST \textit{Attacks}: o atacante procura enviar pedidos HTTP POST que indicam o envio de enormes quantidades de dados de uma só vez, mas, na realidade, este envia quantidades mínimas de dados, de forma espaçada. Mais se acrescenta que estes ataques frequentemente empregam a ferramenta R-U-Dead-Yet que possibilita o envio de dados nos moldes necessários a este tipo de ataque \cite{what_is_rudy}.
    \item \textit{Single Session or Request Attack}: o atacante visa explorar uma vulnerabilidade presente na versão 1.1 do protocolo HTTP, mais especificamente a possibilidade de enviar múltiplos pedidos distintos no \textit{header} de um único pacote.
    \item \textit{Fragmented} HTTP \textit{Flood}: o atacante procura que os \textit{zombies} da sua BOTNET estabelecam conexões HTTP válidas com sistema alvo enviando, posteriormente, pacotes fragmentados de baixa dimensão a uma velocidade curial aos mínimos aceites pelo servidor neste horizonte. Assim, é possível evadir diversas defesas contra ataques DDoS, já que, dadas as características do ataque, a atividade aparenta ser legítima. Deste modo o servidor mantém a sessão ativa e, consequentemente, consome recursos com largura de banda reservada. Mais se acrescenta que este tipo de ataque pode ser concretizado via utilização da ferramenta Slowloris.
    \item \textit{Recursive} GET \textit{Flooding}: o atacante visa sobrecarregar o servidor alvo, mediante o envido de pedidos GET no âmbito da obtenção de longas listas de páginas ou imagens. Assim, este ataque possibilita o consumo de recursos enquanto aparenta ser atividade legítima.
    \item \textit{Random Recursive} GET \textit{Flooding}: o atacante executa um ataque, essencialmente, análogo ao \textit{Recursive} GET \textit{Flooding}, mas emprega uma abordagem que permite selecionar aleatóriamente as páginas requisitadas nos pedidos enviados.
\end{itemize}

No caso concreto do ataque ReDoS, cabe notar que este emprega \textit{regular expressions}, de modo a concretizar um ataque de DDoS. Assim, o atacante constitui pedidos que visam obter quantidades extraordináriamente elevadas e complexas de padrões de pesquisa, possibilitando a exaustão de recursos ou, potencialmente, o acontecimento de \textit{crashes} no sistema alvo \cite{esecurityplanet_types_of_ddos_attacks}.


\subsection{\textit{Ferramentas utilizadas}}
Existem inúmeras ferramentas que possibilitam a concretização de um ataque DDoS, nomeadamente os \textit{stressors}. Neste seguimento, cabe notar que, esta categoria, é constituida por um conjunto extenso de outras ferramentas construídas, de modo a possibilitar a realização de um \textit{stress test} a uma rede. Assim, dadas esta capacidade intrínseca de sobrecarregar uma rede, estas ferramentas podem, também, ser empregues em ataques de DDoS. Mais se acrescenta que, estas ferramentas, possuem níveis de especialidade distintos, já que, algumas, são capazes de executar ataques mediante múltiplos vetores distintos, enquanto que outras especializam-se em concretizar ofensivas numa determinada camada do modelo OSI \cite{cloudflare_ddos_tools, radware_attack_tools}.


Nesta senda, cumpre clarificar algumas categorias de ferramentas empregues em ataques de DDoS, a saber: \textit{Long and slow attack tools}; \textit{Application Layer attack tools}; \textit{Protocol Layer attack tools} \cite{cloudflare_ddos_tools}.

No caso concreto das primeiras ferramentas, tal como o nome indíca, estas proporcionam a possibilidade de realizar ataques DDoS \textit{Long and Slow}, já que a metodologia consite em enviar para o seu alvo uma quantidade baixa de volume de dados a uma velocidade muito lenta. Deste modo, estas ferramentas permite efetuar a manutenção de conexões, em múltiplas portas, com um sistema alvo durante longos períodos de tempo, promovendo um consumo considerável de recursos deste mesmo. Mais se acrescenta que, estas ferramentas, podem, inclusivamente, ser eficientes na concretização de um ataque DDoS, sem a necessidade de utilizar uma BOTNET, sendo frequentemente empregues por atacantes que só possuem acesso a um único dispositivo \cite{cloudflare_ddos_tools, radware_attack_tools}.


Por sua vez, as segundas, executar ataques DDoS de aplicações, mais especificamente ofensivas deste tipo que procuram comprometer a camada 7 do modelo OSI. Assim, um atacante, mediante utilização destas ferramentas e, consequentemente, de uma das técnicas explanada no ponto II.B.3, pode lançar uma quantidade desproporcionada de pedidos a um sistema alvo. Mais se acrescenta que, este tráfego adicional, revela dificuldade de deteção, ou seja, é complicado para as defesas do alvo distinguirem este de pedidos normais, efetuados por utilizados legítimos \cite{cloudflare_ddos_tools}.


As terceiras ferramentas supramencionadas, procuram explorar vulnerabilidades presentes em protocolos, de modo a comprometer a disponibilidade de um sistema alvo. Deste modo, importa realçar que, estas ferramentes, são, habitualmente, empregues por BOTNETs, já que a magnitude do ataque promovido por estas aumenta, considerávelmente, com o acréscimo de quantidade de dispositivos que pertencem a este. Mais se acrescenta que, quando empregues em um único dispositivo, estes utensílios podem ser totalmente ineficiêntes \cite{cloudflare_ddos_tools}.


\hl{TODO: Maybe falar das mais comuns (esta no link da citação)}

\subsection{\textit{Impactos e Motivações}}
Cada ataque DDoS possuí características que o distinguem dos demais, inclusivamente no que aos impactos e motivações diz respeito.


No âmbito dos impactos, cabe notar o prejuízo financeiro associado a um ataque DDoS bem-sucedido a uma organização, já que estes podem causar um decréscimo de produtividade dos funcionários desta mesma, indisponibilidade dos seus serviços, inclusivamente, podendo agravar esta consequencia mediante a potencial perda de clientes e perda global de rendimentos, via incapacidade de realizar os seus processos de negócio. Além disso vão, também, existir custos associados à recuperação dos sistemas afetados, assim como à implementação de medidas que possibilitem precaver e mitigar futuros ataques. Mais se acrescenta que, estes ataques, podem, também, causar complicações legais, mais especificamente disputas entre a organização que sofreu o ataque e os seus clientes, potencialmente agravando, assim, os prejuízo financeiro associado ao ataque \cite{connectwise_types_of_ddos_attacks,cybergc_defending_agaisnt_ddos,stormwall_impacts_ddos}.

Nesta senda, importa, também, realçar que os ataques DDoS podem ser particularmente danificantes para a reputação de uma organização, nomeadamente para empresas que proporcionam serviços onde a disponibilidade revela-se imperativa. Assim, no caso concreto destas empresas, um ataque à disponibilidade dos seus serviços, devido ao \textit{downtime} que causa, pode danificar gravemente a reputação desta mesma, assim como a sua relação com clientes e parceiros. Mais se acrescenta que, um ataque DDoS pode, inclusivamente revelar vulnerabilidades e fraquezas adicionais nos sistemas de uma organização, assim como na sua rede interna que facultam a possibilidade de, posteriormente, serem executados outros tipos de ataques que exploram estes mesmos \cite{connectwise_types_of_ddos_attacks,cybergc_defending_agaisnt_ddos,stormwall_impacts_ddos}.


Por sua vez, no âmbito das motivações, cabe notar que os atacantes justificam as suas ações com base em ganhos financeiros, via utilização de diversas técnicas, nomeadamente extorsão e \textit{blackmailing}, onde são efetuadas promessas de interromper os ataques, mediante o pagamento de um valor de resgate. Além disso, os ataques podem ser perpetrados por um indivíduo que possua a necessidade de vingança para com uma organização específica ou um outro indivíduo, assim como por alguem que procura comprovar as suas capacidades ou que apenas possuí o intuito de obter satisfação pessoal \cite{cybergc_defending_agaisnt_ddos,perimeter81_ddos_motivations}.


Neste seguimento, importa, também, realçar que os ataques DDoS podem ser motivados por ações de \textit{hackvism}, onde um atacante procura utilizar estes ataques para chamar atenção e protestar um político, política ou ideologia, assim como governos e outras instituições relevantes, nomeadamente administrativas ou políticas. Mais se acrescenta que a \textit{cyberwarfare} demonstra ser um assunto progressivamente mais perentório, no horizonte de ataques DDoS, já que estes são crescentemente utilizados por países, de modo a possibilitar o comprometido de diversos aspetos de outros países, permitindo, assim, obter ganhos políticos e militares \cite{cybergc_defending_agaisnt_ddos,perimeter81_ddos_motivations}.